\documentclass[10pt, letterpaper]{article}

% Packages:
\usepackage[
    ignoreheadfoot, % set margins without considering header and footer
    top=2 cm, % seperation between body and page edge from the top
    bottom=2 cm, % seperation between body and page edge from the bottom
    left=2 cm, % seperation between body and page edge from the left
    right=2 cm, % seperation between body and page edge from the right
    footskip=1.0 cm, % seperation between body and footer
    % showframe % for debugging 
]{geometry} % for adjusting page geometry
\usepackage{titlesec} % for customizing section titles
\usepackage{tabularx} % for making tables with fixed width columns
\usepackage{array} % tabularx requires this
\usepackage[dvipsnames]{xcolor} % for coloring text
\definecolor{primaryColor}{RGB}{0, 0, 0} % define primary color
\usepackage{enumitem} % for customizing lists
\usepackage{fontawesome5} % for using icons
\usepackage{amsmath} % for math
\usepackage[
    pdftitle={Manxin Cao's CV},
    pdfauthor={Manxin Cao},
    pdfcreator={LaTeX with RenderCV},
    colorlinks=true,
    urlcolor=primaryColor
]{hyperref} % for links, metadata and bookmarks
\usepackage[pscoord]{eso-pic} % for floating text on the page
\usepackage{calc} % for calculating lengths
\usepackage{bookmark} % for bookmarks
\usepackage{lastpage} % for getting the total number of pages
\usepackage{changepage} % for one column entries (adjustwidth environment)
\usepackage{paracol} % for two and three column entries
\usepackage{ifthen} % for conditional statements
\usepackage{needspace} % for avoiding page brake right after the section title
\usepackage{iftex} % check if engine is pdflatex, xetex or luatex

% Ensure that generate pdf is machine readable/ATS parsable:
\ifPDFTeX
    \input{glyphtounicode}
    \pdfgentounicode=1
    \usepackage[T1]{fontenc}
    \usepackage[utf8]{inputenc}
    \usepackage{lmodern}
\fi

\usepackage{charter}

% Some settings:
\raggedright
\AtBeginEnvironment{adjustwidth}{\partopsep0pt} % remove space before adjustwidth environment
\pagestyle{empty} % no header or footer
\setcounter{secnumdepth}{0} % no section numbering
\setlength{\parindent}{0pt} % no indentation
\setlength{\topskip}{0pt} % no top skip
\setlength{\columnsep}{0.15cm} % set column seperation
\pagenumbering{gobble} % no page numbering

\titleformat{\section}{\needspace{4\baselineskip}\bfseries\large}{}{0pt}{}[\vspace{1pt}\titlerule]

\titlespacing{\section}{
    % left space:
    -1pt
}{
    % top space:
    0.3 cm
}{
    % bottom space:
    0.2 cm
} % section title spacing

\renewcommand\labelitemi{$\vcenter{\hbox{\small$\bullet$}}$} % custom bullet points
\newenvironment{highlights}{
    \begin{itemize}[
        topsep=0.10 cm,
        parsep=0.10 cm,
        partopsep=0pt,
        itemsep=0pt,
        leftmargin=0 cm + 10pt
    ]
}{
    \end{itemize}
} % new environment for highlights


\newenvironment{highlightsforbulletentries}{
    \begin{itemize}[
        topsep=0.10 cm,
        parsep=0.10 cm,
        partopsep=0pt,
        itemsep=0pt,
        leftmargin=10pt
    ]
}{
    \end{itemize}
} % new environment for highlights for bullet entries

\newenvironment{onecolentry}{
    \begin{adjustwidth}{
        0 cm + 0.00001 cm
    }{
        0 cm + 0.00001 cm
    }
}{
    \end{adjustwidth}
} % new environment for one column entries

\newenvironment{twocolentry}[2][]{
    \onecolentry
    \def\secondColumn{#2}
    \setcolumnwidth{\fill, 4.5 cm}
    \begin{paracol}{2}
}{
    \switchcolumn \raggedleft \secondColumn
    \end{paracol}
    \endonecolentry
} % new environment for two column entries

\newenvironment{threecolentry}[3][]{
    \onecolentry
    \def\thirdColumn{#3}
    \setcolumnwidth{, \fill, 4.5 cm}
    \begin{paracol}{3}
    {\raggedright #2} \switchcolumn
}{
    \switchcolumn \raggedleft \thirdColumn
    \end{paracol}
    \endonecolentry
} % new environment for three column entries

\newenvironment{header}{
    \setlength{\topsep}{0pt}\par\kern\topsep\centering\linespread{1.5}
}{
    \par\kern\topsep
} % new environment for the header

\newcommand{\placelastupdatedtext}{% \placetextbox{<horizontal pos>}{<vertical pos>}{<stuff>}
  \AddToShipoutPictureFG*{% Add <stuff> to current page foreground
    \put(
        \LenToUnit{\paperwidth-2 cm-0 cm+0.05cm},
        \LenToUnit{\paperheight-1.0 cm}
    ){\vtop{{\null}\makebox[0pt][c]{
        \small\color{gray}\textit{Last updated in July 2024}\hspace{\widthof{Last updated in July 2024}}
    }}}%
  }%
}%

% save the original href command in a new command:
\let\hrefWithoutArrow\href

% new command for external links:


\begin{document}
    \newcommand{\AND}{\unskip
        \cleaders\copy\ANDbox\hskip\wd\ANDbox
        \ignorespaces
    }
    \newsavebox\ANDbox
    \sbox\ANDbox{$|$}

    \begin{header}
        \fontsize{25 pt}{25 pt}\selectfont Manxin Cao

        \vspace{5 pt}

        \normalsize
        \kern 5.0 pt%
        \mbox{\hrefWithoutArrow{caomanxin@gmail.com}{caomanxin@gmail.com}}%
        \kern 5.0 pt%
        \AND%
        \kern 5.0 pt%
        \mbox{\hrefWithoutArrow{tel: +1 8052841589}{+1 8052841589}}%
        \kern 5.0 pt%
        \AND%
        \kern 5.0 pt%
        \mbox{\hrefWithoutArrow{https://manxin-c.github.io/}{https://manxin-c.github.io/}}%
 
    \end{header}

    \vspace{5 pt - 0.3 cm}


    

    \section{EDUCATION}



        
        \begin{twocolentry}{
            \normalsize Sept. 2021 - Jun. 2025
        }{
           {\setlength{\fontsize}{11pt}{14pt}\selectfont \textbf{Inner Mongolia University}, Inner Mongolia, China}}
            \item Bachelor of Biotechnology\end{twocolentry}
        \vspace{0.10 cm}
        \begin{onecolentry}
            \begin{highlights}
                \item GPA: \textbf{3.87/4.0} | Average Score: \textbf{90.60} | Ranking: \textbf{1/81} (1.2\%)  (\href{https://manxin-c.github.io/assets/Transcript.pdf}{Transcript})
                \item \textbf{Core courses:} Cell Biology (93) | Biochemistry (92) | General Biology (98) | Genetic Engineering A (92) | Genetics (90)
                \item \textbf{Basic courses:} College Physics B1 \& B2 (96) | Organic Chemistry (96) | Advanced Mathematics B1 (93) | Advanced Mathematics B2 (91) | Enzyme Engineering (92)


            \end{highlights}
        \end{onecolentry}




\section{SKILLS}




       \begin{onecolentry}
            \textbf{Molecular Biology \& Biochemistry:} PCR, Agarose Gel Electrophoresis, SDS-PAGE, Nickel Affinity Gravity Column Chromatography, Flag-tag Protein Purification, FPLC, Size Exclusion Chromatography, Western Blotting, AlphaLISA Assay, Methylation Activity Assay, Fluorescence Anisotropy 
        \end{onecolentry}
        
       \vspace{0.2 cm}

       \begin{onecolentry}
            \textbf{Structural Biology:} Preliminary Screening for X-ray Crystallization Conditions
        \end{onecolentry}
        
       \vspace{0.2 cm}

 \begin{onecolentry}
            \textbf{Observed Techniques:} Transfection, SPR, Flow Cytometry, Negative Staining, X-ray Crystallography, Cryo-EM
        \end{onecolentry}
        
       \vspace{0.2 cm}

        \begin{onecolentry}
            \textbf{Computational:} Python(basic), R (basic)
        \end{onecolentry}

        \vspace{0.2 cm}

        \begin{onecolentry}
            \textbf{Language:} Native in Mandarin Chinese, Proficient in English (TOEFL: 100), Intermediate in French
        \end{onecolentry}

      

    

    \section{EXCHANGE EXPERIENCE}



        
      \begin{twocolentry}{
            \normalsize Sept. 2023 - Jan. 2024
        }{
           {\setlength{\fontsize}{10.7pt}{14pt}\selectfont \textbf{Peking University, Peking, China}}}\end{twocolentry}
\vspace{0.1 cm}
           \normalsize {Prof. Su, Xiaodong's lab}\\
\vspace{0.1 cm}
        \emph{Exchange Student Researcher, Advisor: Professor Su, Xiaodong}
\vspace{0.1 cm}
        \begin{onecolentry}
            \begin{highlights}
                \item Assisted in elucidating a protein's structure to design its inhibitors. Performed numerous transformations, protein expression, and purification using nickel affinity gravity columns and size exclusion chromatography to obtain stable proteins with correct structures suitable for X-ray analysis. Additionally, I also conducted preliminary screening for X-ray crystallization.
                \item Proficiently performed various related operations throughout the process, such as plasmid extraction, gel extraction, agarose gel electrophoresis, SDS-PAGE and more.
                \item Shadowed graduate students to gain exposure to a variety of techniques, including structural analysis methods such as crystal preparation and X-ray diffraction. Also observed and gained foundational insights into methodologies such as flow cytometry, negative staining, Surface Plasmon Resonance (SPR), and transfection.
                \item \textbf{Note:} Due to the confidentiality of the project, specific details cannot be disclosed until publication.
            \end{highlights}
        \end{onecolentry}



    
    \section{PROJECTS}



        \begin{onecolentry}
        \setlength{\fontsize}{10.5pt}{14pt}\selectfont \textbf{Project1: Construction of Engineering E. coli for Sustainable Straw Biodegradation}\end{onecolentry}   
\vspace{0.1 cm}    
        \begin{twocolentry}{
            \normalsize IMU, Inner Mongolia, China
        }{Prof. Mo, Rigen's lab
            }
            \end{twocolentry}
\vspace{0.1 cm}
        \begin{twocolentry}{
            \normalsize Mar. 2023 - Present
        }{
            {\emph{Project Lead, Advisor: Professor Mo, Rigen}}} \end{twocolentry}
   
        \vspace{0.10 cm}
        \begin{onecolentry}
            \begin{highlights}
                \item Cloned the cellulase gene, manganese peroxidase gene, lignin peroxidase gene, and laccase gene from Trichoderma reesei and white rot fungi and transferred them into $\Delta mrdA$-Escherichia coli, which has strong cell wall permeability. 
                \item Improved conditions for high-level expression and extracellular secretion of target proteins in engineered E. coli, increasing straw degradation potential as detected by Fourier-transform infrared spectroscopy.
                \item Completed the project defense and mid-term defense, leading a team of five. The project work was recognized with the second prize at the China International College Students Innovation Competition at the university level.
            \end{highlights}
        \end{onecolentry}


        \vspace{0.2 cm}

       \begin{onecolentry}
       \setlength{\fontsize}{10.5pt}{14pt}\selectfont \textbf{Project2: Structural Insights into Brucella abortus Succinate Dehydrogenase for Combating Brucellosis}\end{onecolentry}
\vspace{0.1 cm}
            \begin{twocolentry}{
            \normalsize IMU, Inner Mongolia, China
        }{Lecturer Zhou, Xiaoting's lab
            }
           \end{twocolentry}
\vspace{0.1 cm}
        \begin{twocolentry}{
            \normalsize Mar. 2024 - Present
        }{
            {\emph{Project Co-Lead, Advisor: Lecturer Zhou, Xiaoting}}} \end{twocolentry}

        \vspace{0.10 cm}
        \begin{onecolentry}
            \begin{highlights}
                \item Optimized the expression and purification processes of succinate dehydrogenase to ensure proportional expression of all four subunits, resulting in a well-structured and functional tetramer.
                \item Performed skillfully in primer design, PCR, transformation, expression, purification using nickel columns and size exclusion chromatography (SEC), SDS-PAGE, and Western blotting.
                \item Proficiency in experimental workflow analysis, with the autonomy to independently adjust various parameters such as buffer concentrations, components, and conditions, and autonomously adaptable in refining experimental plans based on outcomes, including troubleshooting and optimizing protocols for success.
            \end{highlights}
        \end{onecolentry}


        \vspace{0.2 cm}

        \begin{onecolentry}
        \setlength{\fontsize}{10.5pt}{14pt}\selectfont \textbf{Project3: Characterization of DNMT3A Mutants in Binding to Histone Modifications for Mapping Unknown Interaction Sites}
        \end{onecolentry}
\vspace{0.1 cm}
        \begin{twocolentry}{
            \normalsize UCSB, Santa Barbara, CA
        }{Distinguished Professor Norbert Reich's lab
             }
             \end{twocolentry}
\vspace{0.1 cm}
        \begin{twocolentry}{
            \normalsize Jul. 2024 - Oct. 2024
        }{
           {\emph{Summer Research Scholar, Advisor: Prof. Norbert Reich}}} \end{twocolentry}

        \vspace{0.10 cm}
        \begin{onecolentry}
            \begin{highlights}
                \item Constructed and purified DNMT3A mutants across various domains using fast protein liquid chromatography (FPLC) with nickel column and size exclusion chromatography (SEC), achieving a high yield and purity of mutant stocks.
                \item Performed AlphaLISA binding assays and Fluorescence anisotropy with DNMT3A mutants, systematically refining buffer conditions to minimize signals arising from non-specific binding.
                \item Working on constructing a truncated version of DNMT3A, devoid of a specific domain, to produce more reliable proteins for use in AlphaLISA binding assays.
            \end{highlights}
        \end{onecolentry}



    
    \section{HONORS AND SCHOLARSHIPS}



        
        \begin{onecolentry}
            {\setlength{\fontsize}{11pt}{14pt}\selectfont \textbf{Ministry of Education}}  \end{onecolentry}
 \vspace{0.1 cm}
            \begin{twocolentry}{
            \normalsize October 2023
        }
        \normalsize National Scholarship
        \end{twocolentry}

        \vspace{0.2 cm}

        \begin{onecolentry}
            {\setlength{\fontsize}{11pt}{14pt}\selectfont \textbf{Inner Mongolia University}}  \end{onecolentry}
 \vspace{0.1 cm}
            \begin{twocolentry}{
            \normalsize April 2023 \& April 2024
        }
        \normalsize The First Prize Scholarship
        \end{twocolentry}
 \vspace{0.1 cm}
            \begin{twocolentry}{
            \normalsize May 2023 \& April 2024
        }
        \normalsize Merit Student
        \end{twocolentry}
 \vspace{0.1 cm}
            \begin{twocolentry}{
            \normalsize May 2024
        }
        \normalsize Model Student Scholarship
        \end{twocolentry}
 

\end{document}